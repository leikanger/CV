%
% Notes:
%
% * Don't forget to change `pdfauthor' and `keywords' in the \hypersetup
%   section below.
%
% * To create a new page use: \newpage \opening
%
% * res.cls includes an \address{} command which can be used up to twice,
%   but my address is too long for the format it uses.
%
% * Alternate documentclass statement to put headings in margin:
%   \documentclass[margin,line,11pt,final]{res}
%
% * Can divide publication/presentation list into subsections by year:
%   \section{\bf\small\hspace{8mm}2006}
%
%%----------------------------------------------------------------------------%


%\documentclass[a4paper, norsk, 11pt]{article}

\documentclass[overlapped, norsk, 11pt, line,letterpaper]{res}

%start per
\usepackage[T1]{fontenc}    %ndvendig til fonter
\usepackage[latin1]{inputenc}
%\usepackage{babel}
%slutt per

\usepackage{ifpdf}

\ifpdf
  \usepackage[pdftex]{hyperref}
\else
  \usepackage[hypertex]{hyperref}
\fi

\hypersetup{
  letterpaper,
  colorlinks,
  urlcolor=black,
  pdfpagemode=none,
  pdftitle={Curriculum Vitae},
  pdfauthor={Per R. Leikanger},
  pdfcreator={$ $Id: cv-us.tex,v 1.28 2006/12/12 22:53:52 jrblevin Exp $ $},
  pdfsubject={Curriculum Vitae},
  pdfkeywords={CV, autonomy, programming, computer science, control, cybernetics, c++, linux, unix}
}

%%===========================================================================%%

\begin{document}

%---------------------------------------------------------------------------
% Document Specific Customizations

% Make lists without bullets and with no indentation
\setlength{\leftmargini}{0em}
\renewcommand{\labelitemi}{}

% Use large bold font for printed name at top of pages
\renewcommand{\namefont}{\large\textbf}

%---------------------------------------------------------------------------

\name{Per R. Leikanger}

\begin{resume}

\begin{ncolumn}{2}
	Per Roald Leikanger 				& 	tel. +47 400 40 814 		\\
	Engesetgeila 10	    				&	{\tt leikange@gmail.com}		\\
	N-6100 Volda                       &	{\tt github.com/leikanger} \\
	Norway  					  	&  \\
\end{ncolumn}

%---------------------------------------------------------------------------

\section{\bf Education}
PhD Autonomous Navigation, UiT, 2022    (in review)

MSc Engineering Cybernetics, NTNU, 2012

All available courses from the Kavli Institute of Systems Neuroscience/CBM, NTNU, 2011
 
%---------------------------------------------------------------------------
\section{\bf Academic Vitae}
PhD Candidate, Autonomous Navigation in (the animal and) the Machine, UiT, 2017 - 2022

Visiting Scholar of \emph{Richard Sutton}, RLAI lab/AMII, Edmonton, 2020

    \emph{\small Project Summary:} 
    {\small
    Reinforcement Learning (RL) is the branch of AI capable of interaction learning and autonomous control,
    however, 
    today's methods are limited for large problems or real-world applications.
    In my PhD, I combined knowledge on neural navigation from the Moser lab with Richard Sutton's insight into modern RL.
    %My PhD project combined ground-breaking research from the Moser lab with Richard Sutton's insight on modern RL.
    %My work combines ground-breaking research from the Moser lab with Richard Sutton's insight on modern RL.
    Results are ready to be implemented on real hardware. See \href{www.neoRL.net}{\underline{www.neoRL.net}}
    }

%\href{https://ccneuro.org/2019/proceedings/0001079.pdf}{\underline{Publication}}:
%  Modular RL for real-time learning {\tiny in physical environments}
%  CCN, Berlin, 2019 
  
%---------------------------------------------------------------------------

\section{\bf Positions }



\begin{format}
\title{l}\dates{r}\\
\employer{l}\location{r}\\
\body\\
\end{format}

\title{PhD Candidate}
\employer{\emph{UiT - the Arctic University of Norway}}
\location{\emph{Troms�}}
\dates{2017 - present}
\begin{position}
    %Autonomous navigation in (the animal and) the machine: 
    Combining inspiration from neural representation of space, results from the Moser lab resulting in the 2014 Nobel price, with RL in AI, %, as pioneered by Richard Sutton, 
        the neoRL approach have been implemented and shown capable of real-time autonomous navigation.
    The neoRL approach is rooted in the superposition principle, and experiments show neoRL to be ready for maritime autonomy.
    % The idea has been implemented and demonstrated for general 2D Euclidean spaces, and is ready for maritime autonomy. 
    %A proof-of-concept have been implemented and results published in renowned conferences; neoRL is ready for demonstrating tomorrow's maritime autonomy.
%%%%%
  % Maritime autonomy enabled by interaction learning;
  % Learning by interaction for real ships if challenging for today's
  % reinforcement learning.
  % I believe this is possible by taking inspiration from 
  % orientation in neuroscience (the focus of Moser lab) 
  % and linear systems' theory from cybernetics, which has come to define my PhD:
  % Fundamental RL research.
\end{position}

\title{Inventor and Founder}
\employer{\emph{Eupnea}}
\location{\emph{Troms�}}
\dates{January 2016 - 2017}
\begin{position}
  Automatic instrumentation and reporting of respiration frequency,
  allowing for continuous feedback to the attending physician.
  Eupnea is still running, currently with 6 employees.
\end{position}

\title{Head of R\&D}
\employer{\emph{Breivoll Inspection Technologies AS}}
\location{\emph{Troms�}}
\dates{January 2015 - June 2016}
\begin{position}
  Responsible for R\&D on signal acquisition and ultrasound imaging an PS3 pipescanner.
  Organization of international endeavours and collaboration with Norwegian universities and NDT professionals. 
\end{position}

\title{Senior Software Developer}
\employer{\emph{Rolls-Royce Marine}}
    \location{\emph{�lesund}}
\dates{June 2014 - December 2014}
\begin{position}
	Programming in C++ for \emph{Rolls-Royce Marine}, working on \emph{Integrated Bridge}. 
\end{position}

%\newpage
\title{R\&D engineer}
\employer{\emph{Archer -- the Well Company}}
\location{\emph{Bergen}}
%\dates{30.10.2012 - present}
\dates{October 2012 - Mai 2014}
\begin{position}
    Research on signal processing and development of Archer's tool for ultrasound imaging.
    % Research and development of Archer's tool for 
	% Research on the properties of sound from turbulent current under water and
        %active ultrasound for location of leaks and integrity analysis of pipes.
	Research and development of Archer's tool park for well integrity analysis.
    Ultrasound imaging.
\end{position}

\title{Teaching Assistant} %, ``Real-Time Programming''}
\employer{\emph{NTNU}}
    \location{\emph{Trondheim}}
\dates{2009}
\begin{position}
	Teaching assistant in \emph{TTK4145 -- Real-time programming}, a course about design, analysis and programming of a system's real-time properties.
%	The position was about helping students through the course's set of programming exercises.
	The language used was \emph{C} with \emph{Posix}.
\end{position}

\title{Recruiting Employee}
\employer{\emph{NTNU / Studiestart A/S}}
    \location{\emph{Trondheim}}
\dates{2008}
\begin{position}
	NTNU's representative on Education \& Recruitment Day in four provinces of Norway.
	During the period from january to mars 2008, I was NTNU's representative on ``Studie og yrkesorienteringsdag'' at high scools in Nord-Tr�ndelag, S�r-Tr�ndelag, Oppland and Hedemark.
	A series of four different lectures about studying at NTNU, each on half an hour, was held on most of the colleges and high scools in the mentioned provinces. 
	% Skrive at det var til saman 42 skoler?
\end{position}

\title{Developer}
\employer{Industrial Control Design A/S}
\location{\emph{�lesund}}
\dates{\emph{2007}}
\begin{position}
	Summer position as developer for ICD.
	Development and testing of general and standarized filters and control components to be delivered in a standard package with Control Design Platform, ICD's main product.
	Programming in C++, testing in WinXP, linux.% and RTOS-32.
\end{position}

\title{Teaching Assistant} %, ``Computerized Control for Industrial Systems''}
\employer{\emph{NTNU}}
\location{\emph{Trondheim}}
\dates{2007}
\begin{position}
	Teaching assistant in \emph{TTK4125 -- Computerized Control in Industrial Systems}.
	Helping the students through a set of excercises and labs about C programming and low-level programming for avr.
	Measurement principles.
\end{position}

\section{\bf Voluntary Work}

\title{Volunteer}
\employer{\emph{Fotogjengen, Studentersamfundet}}
\location{\emph{Trondheim}}
\dates{2004 - 2007}
\begin{position}
	``Funksjon�r'' at the photo group in the Student Community in Trondheim, involving approximately 20 hours of work per week during the contract time of 2.5 years.\\
	\emph{2005:} main responsibility of FG's IT systems. Operation and maintainance of a sales site for FG's pictures with more than 350.000 entries. \emph{www.fotogjengen.no}\\
	\emph{2006:} Operational procurement and maintenance for FG's darkroom and photo plotter.
\end{position}

% \title{Board member/ Web master}
% \employer{\emph{Excursion Committe, Kyb09}}
%     \location{\emph{Trondheim}}
% \dates{2007}
% \begin{position}
% 	Planning and collection of funds for 3. class cybernetics' excursion to Los Angeles, spring 2007.
% \end{position}

\title{Volunteer}
\employer{\emph{UKA05, isfit05, UKA07, isfit07, UKA09}}
\location{\emph{Trondheim}}
\dates{2005, 2007, 2009}
\begin{position}
	Volunteer as photographer through \emph{Fotogjengen}.
\end{position}

% \title{Volunteer}
% \employer{\emph{Isfit -05, Isfit -07}}
% \location{\emph{Trondheim}}
% \dates{2005, 2007}
% \begin{position}
% 	Volunteer as photographer through \emph{Fotogjengen}.
% \end{position}

%\title{Chairman}
%\employer{\emph{Utr�na Seglarlag}}
%\location{\emph{Volda}}
%\dates{2003 - 2004}
%\begin{position}
%	Chairmain for \emph{Utr�na brettseglarlag} in Volda -- Sports club for windsurfing and kitesurfing in Volda, Sunnm�re.
%\end{position}


% \title{Data puncher}
% \employer{DMF, NTNU/Helse Sunnm�re}
% \location{Volda}
% \dates{2003 - 2004}
% \begin{position}
% 	Registering the information from questionnaires in a large research project on anxiety amongst children and adolecents.
% \end{position}
%---------------------------------------------------------------------------

%\section{\bf References}
%
%\begin{tabular}
%\end{tabular}


%\section{References}
%\begin{tabular}{l l l}
%Amund Skavhaug & Supervisor on MSc project & +47 73 59 4396 \\
%\emph{amund.skavhaug@ntnu.no} &&
%\end{tabular}


% \section{\bf Service}
% 
% \begin{itemize}
% \item Vice President, Society of Undergraduate Mathematics, North
%   Carolina State University, 2001--2003
% \item Vice President, Economics Graduate Student Council, Duke
%   University, 2006--2007
% \end{itemize}
% %%---------------------------------------------------------------------------%%

%\begin{center}
%\vspace{\fill}\ \newline
%{\tiny \rm $ $RCSfile: cv-us.tex,v $ $ }
%{\tiny \rm $ $Date: \today $ $ }
%{\tiny \rm $ $Revision: 1.0 $ $ }
%\end{center}

\end{resume}

\end{document}

%%===========================================================================%%
