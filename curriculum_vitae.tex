%
% Notes:
%
% * Don't forget to change `pdfauthor' and `keywords' in the \hypersetup
%   section below.
%
% * To create a new page use: \newpage \opening
%
% * res.cls includes an \address{} command which can be used up to twice,
%   but my address is too long for the format it uses.
%
% * Alternate documentclass statement to put headings in margin:
%   \documentclass[margin,line,11pt,final]{res}
%
% * Can divide publication/presentation list into subsections by year:
%   \section{\bf\small\hspace{8mm}2006}
%
%%----------------------------------------------------------------------------%


%\documentclass[a4paper, norsk, 11pt]{article}

\documentclass[overlapped, norsk, 11pt, line,letterpaper]{res}

%start per
\usepackage[T1]{fontenc}    %ndvendig til fonter
\usepackage[latin1]{inputenc}
%\usepackage{babel}
%slutt per

\usepackage{ifpdf}

\ifpdf
  \usepackage[pdftex]{hyperref}
\else
  \usepackage[hypertex]{hyperref}
\fi

\hypersetup{
  letterpaper,
  colorlinks,
  urlcolor=black,
  pdfpagemode=none,
  pdftitle={Curriculum Vitae},
  pdfauthor={Per R. Leikanger},
  pdfcreator={$ $Id: cv-us.tex,v 1.28 2006/12/12 22:53:52 jrblevin Exp $ $},
  pdfsubject={Curriculum Vitae},
  pdfkeywords={matematikk differensialligninger linear algebra regulering linux unix c c++}
}

%%===========================================================================%%

\begin{document}

%---------------------------------------------------------------------------
% Document Specific Customizations

% Make lists without bullets and with no indentation
\setlength{\leftmargini}{0em}
\renewcommand{\labelitemi}{}

% Use large bold font for printed name at top of pages
\renewcommand{\namefont}{\large\textbf}

%---------------------------------------------------------------------------

\name{Per R. Leikanger}

\begin{resume}

\begin{ncolumn}{2}
	Per Roald Leikanger 				& 	tel. +47 400 40 814 		\\
	Engesetgeila 10	    				&	{\tt leikange@gmail.com}		\\
	N-6100 Volda                       &	{\tt github.com/leikanger} \\
	Norway  					  	&  \\
\end{ncolumn}

%---------------------------------------------------------------------------

\section{\bf Education}

MSc Engineering Cybernetics, NTNU, 2012

% ABC for Prosjektstyring, Course held by Tekna, 2013

All available courses from the Kavli Institute of Systems Neuroscience/CBM, NTNU, 2011

% Presentation Course by Karl A. Olsen, Studieavdelingen NTNU, 2008
 
%---------------------------------------------------------------------------
\section{\bf Academic Vitae}
Visiting Scholar at UoA RLAI lab with Richard Sutton, Edmonton, 2020

\href{https://ccneuro.org/2019/proceedings/0001079.pdf}{\underline{Publication}}:
  Modular RL for real-time learning {\tiny in physical environments}
  CCN, Berlin, 2019 
  
PhD Candidate, Reinforcement Learning for Ship Autonomy, UiT, 2017 - present

%---------------------------------------------------------------------------

\section{\bf Positions }



\begin{format}
\title{l}\dates{r}\\
\employer{l}\location{r}\\
\body\\
\end{format}

\title{PhD Candidate}
\employer{\emph{UiT - Arctic University of Norway}}
\location{\emph{Troms�}}
\dates{2017 - present}
\begin{position}
  Maritime autonomy enabled by interaction learning;
  Learning by interaction for real ships if challenging for today's
  reinforcement learning.
%   Learning by interaction to enable maritime autonomy, i.e. learning for real
%   ships, is challenging for today's reinforcement learning.
  I believe this is possible by taking inspiration from 
  orientation in neuroscience (the focus of Moser lab) 
  and linear systems' theory from cybernetics, which has come to define my PhD:
  Fundamental RL research.
%   %Reactive path planning and control for physical ships is challenging for   today's approach to RL. 
%   Utilize theory from cybernetics and neuroscience to develop reinforcement
%   learning to a level where real-time learning is possible;
%   for safe and efficient real-world interaction learning.
% %  Utilize theory from cybernetics and neuroscience to develop reinforcement
% %  learning to the necessary sample efficiency for real-world control.
%   Maritime autonomy.
\end{position}

\title{Inventor and Founder}
\employer{\emph{Eupnea}}
\location{\emph{Troms�}}
\dates{January 2016 - 2017}
\begin{position}
  Automatic instrumentation and reporting of one of the important
  vital parameters, giving continuous feedback to the attending physician
  about the patients condition.  
  %We have started with the next functional prototype for automatic sensing of one of four vital parameters, giving better feedback to the attending doctor about the immediate condition of the patient. 
  %Part of SIVA's incubation programme at Forskningsparken.
  Embedded development, instumentation, Linux, BOOST, software
  development. Project management.
\end{position}

\title{Head of R\&D}
\employer{\emph{Breivoll Inspection Technologies AS}}
\location{\emph{Troms�}}
\dates{January 2015 - June 2016}
\begin{position}
  Responsible for signal acquisition and processing for the newly developed PS3
  pipe scanner. Planning of the next generation pipescanner (PS4). 
  Organizing and distributing resources related to development: DataRespons in
  Norway, R\&D department in Moskow, and collaborating Norwegian universities and NDT professinals.
\end{position}

\title{Senior Software Developer}
\employer{\emph{Rolls-Royce Marine}}
    \location{\emph{�lesund}}
\dates{June 2014 - December 2014}
\begin{position}
	Programming in C++ for \emph{Rolls-Royce Marine}, working on \emph{Integrated Bridge}. 
\end{position}

\title{R\&D engineer}
\employer{\emph{Archer -- the Well Company}}
\location{\emph{Bergen}}
%\dates{30.10.2012 - present}
\dates{October 2012 - Mai 2014}
\begin{position}
	Research on the properties of sound from turbulent current under water and
  active ultrasound for location of leaks and integrity analysis of pipes.
	Research and development of Archer's tool park for well integrity analysis.
  Ultrasound imaging.
\end{position}

\title{Teaching Assistant, ``Real-Time Programming''}
\employer{\emph{NTNU}}
    \location{\emph{Trondheim}}
\dates{2009}
\begin{position}
	Teaching assistant in \emph{TTK4145 -- Real-time programming}, a course about design, analysis and programming of a system's real-time properties.
%	The position was about helping students through the course's set of programming exercises.
	The language used was \emph{C} with \emph{Posix}.
\end{position}

\title{Recruiting Employee}
\employer{\emph{NTNU / Studiestart A/S}}
    \location{\emph{Trondheim}}
\dates{2008}
\begin{position}
	NTNU's representative on Education \& Recruitment Day in four provinces of Norway.
	During the period from january to mars 2008, I was NTNU's representative on ``Studie og yrkesorienteringsdag'' at high scools in Nord-Tr�ndelag, S�r-Tr�ndelag, Oppland and Hedemark.
	A series of four different lectures about studying at NTNU, each on half an hour, was held on most of the colleges and high scools in the mentioned provinces. 
	% Skrive at det var til saman 42 skoler?
\end{position}

\title{Developer}
\employer{Industrial Control Design A/S}
\location{\emph{�lesund}}
\dates{\emph{2007}}
\begin{position}
	Summer position as developer for ICD.
	Development and testing of general and standarized filters and control components to be delivered in a standard package with Control Design Platform, ICD's main product.
	Programming in C++, testing in WinXP, linux.% and RTOS-32.
\end{position}

\title{Teaching Assistant, ``Computerized Control in Industrial Systems''}
\employer{\emph{NTNU}}
\location{\emph{Trondheim}}
\dates{2007}
\begin{position}
	Teaching assistant in \emph{TTK4125 -- Computerized Control in Industrial Systems}.
	Helping the students through a set of excercises and labs about C programming and low-level programming for avr.
	Measurement principles.
	% TA MED LINK TIL FAGET! XXX
\end{position}

\section{\bf Voluntary Work}

\title{Volunteer}
\employer{\emph{Fotogjengen, Studentersamfundet}}
\location{\emph{Trondheim}}
\dates{2004 - 2007}
\begin{position}
	``Funksjon�r'' at the photo group in the Student Community in Trondheim, involving approximately 20 hours of work per week during the contract time of 2.5 years.\\
	\emph{2005:} main responsibility of FG's IT systems. Operation and maintainance of a sales site for FG's pictures with more than 350.000 entries. \emph{www.fotogjengen.no}\\
	\emph{2006:} Operational procurement and maintenance for FG's darkroom and photo plotter.
\end{position}

% \title{Board member/ Web master}
% \employer{\emph{Excursion Committe, Kyb09}}
%     \location{\emph{Trondheim}}
% \dates{2007}
% \begin{position}
% 	Planning and collection of funds for 3. class cybernetics' excursion to Los Angeles, spring 2007.
% \end{position}

\title{Volunteer}
\employer{\emph{UKA05, UKA07, UKA09}}
\location{\emph{Trondheim}}
\dates{2005, 2007, 2009}
\begin{position}
	Volunteer as photographer through \emph{Fotogjengen}.
\end{position}

% \title{Volunteer}
% \employer{\emph{Isfit -05, Isfit -07}}
% \location{\emph{Trondheim}}
% \dates{2005, 2007}
% \begin{position}
% 	Volunteer as photographer through \emph{Fotogjengen}.
% \end{position}

\title{Chairman}
\employer{\emph{Utr�na Seglarlag}}
\location{\emph{Volda}}
\dates{2003 - 2004}
\begin{position}
	Chairmain for \emph{Utr�na brettseglarlag} in Volda -- Sports club for windsurfing and kitesurfing in Volda, Sunnm�re.
\end{position}


% \title{Data puncher}
% \employer{DMF, NTNU/Helse Sunnm�re}
% \location{Volda}
% \dates{2003 - 2004}
% \begin{position}
% 	Registering the information from questionnaires in a large research project on anxiety amongst children and adolecents.
% \end{position}
%---------------------------------------------------------------------------

%\section{\bf References}
%
%\begin{tabular}
%\end{tabular}


%\section{References}
%\begin{tabular}{l l l}
%Amund Skavhaug & Supervisor on MSc project & +47 73 59 4396 \\
%\emph{amund.skavhaug@ntnu.no} &&
%\end{tabular}


% \section{\bf Service}
% 
% \begin{itemize}
% \item Vice President, Society of Undergraduate Mathematics, North
%   Carolina State University, 2001--2003
% \item Vice President, Economics Graduate Student Council, Duke
%   University, 2006--2007
% \end{itemize}
% %%---------------------------------------------------------------------------%%

%\begin{center}
%\vspace{\fill}\ \newline
%{\tiny \rm $ $RCSfile: cv-us.tex,v $ $ }
%{\tiny \rm $ $Date: \today $ $ }
%{\tiny \rm $ $Revision: 1.0 $ $ }
%\end{center}

\end{resume}

\end{document}

%%===========================================================================%%
